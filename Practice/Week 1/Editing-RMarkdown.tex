% Options for packages loaded elsewhere
% Options for packages loaded elsewhere
\PassOptionsToPackage{unicode}{hyperref}
\PassOptionsToPackage{hyphens}{url}
\PassOptionsToPackage{dvipsnames,svgnames,x11names}{xcolor}
%
\documentclass[
  12pt,
  letterpaper,
  DIV=11,
  numbers=noendperiod]{scrartcl}
\usepackage{xcolor}
\usepackage{amsmath,amssymb}
\setcounter{secnumdepth}{-\maxdimen} % remove section numbering
\usepackage{iftex}
\ifPDFTeX
  \usepackage[T1]{fontenc}
  \usepackage[utf8]{inputenc}
  \usepackage{textcomp} % provide euro and other symbols
\else % if luatex or xetex
  \usepackage{unicode-math} % this also loads fontspec
  \defaultfontfeatures{Scale=MatchLowercase}
  \defaultfontfeatures[\rmfamily]{Ligatures=TeX,Scale=1}
\fi
\usepackage{lmodern}
\ifPDFTeX\else
  % xetex/luatex font selection
\fi
% Use upquote if available, for straight quotes in verbatim environments
\IfFileExists{upquote.sty}{\usepackage{upquote}}{}
\IfFileExists{microtype.sty}{% use microtype if available
  \usepackage[]{microtype}
  \UseMicrotypeSet[protrusion]{basicmath} % disable protrusion for tt fonts
}{}
\makeatletter
\@ifundefined{KOMAClassName}{% if non-KOMA class
  \IfFileExists{parskip.sty}{%
    \usepackage{parskip}
  }{% else
    \setlength{\parindent}{0pt}
    \setlength{\parskip}{6pt plus 2pt minus 1pt}}
}{% if KOMA class
  \KOMAoptions{parskip=half}}
\makeatother
% Make \paragraph and \subparagraph free-standing
\makeatletter
\ifx\paragraph\undefined\else
  \let\oldparagraph\paragraph
  \renewcommand{\paragraph}{
    \@ifstar
      \xxxParagraphStar
      \xxxParagraphNoStar
  }
  \newcommand{\xxxParagraphStar}[1]{\oldparagraph*{#1}\mbox{}}
  \newcommand{\xxxParagraphNoStar}[1]{\oldparagraph{#1}\mbox{}}
\fi
\ifx\subparagraph\undefined\else
  \let\oldsubparagraph\subparagraph
  \renewcommand{\subparagraph}{
    \@ifstar
      \xxxSubParagraphStar
      \xxxSubParagraphNoStar
  }
  \newcommand{\xxxSubParagraphStar}[1]{\oldsubparagraph*{#1}\mbox{}}
  \newcommand{\xxxSubParagraphNoStar}[1]{\oldsubparagraph{#1}\mbox{}}
\fi
\makeatother

\usepackage{color}
\usepackage{fancyvrb}
\newcommand{\VerbBar}{|}
\newcommand{\VERB}{\Verb[commandchars=\\\{\}]}
\DefineVerbatimEnvironment{Highlighting}{Verbatim}{commandchars=\\\{\}}
% Add ',fontsize=\small' for more characters per line
\usepackage{framed}
\definecolor{shadecolor}{RGB}{241,243,245}
\newenvironment{Shaded}{\begin{snugshade}}{\end{snugshade}}
\newcommand{\AlertTok}[1]{\textcolor[rgb]{0.68,0.00,0.00}{#1}}
\newcommand{\AnnotationTok}[1]{\textcolor[rgb]{0.37,0.37,0.37}{#1}}
\newcommand{\AttributeTok}[1]{\textcolor[rgb]{0.40,0.45,0.13}{#1}}
\newcommand{\BaseNTok}[1]{\textcolor[rgb]{0.68,0.00,0.00}{#1}}
\newcommand{\BuiltInTok}[1]{\textcolor[rgb]{0.00,0.23,0.31}{#1}}
\newcommand{\CharTok}[1]{\textcolor[rgb]{0.13,0.47,0.30}{#1}}
\newcommand{\CommentTok}[1]{\textcolor[rgb]{0.37,0.37,0.37}{#1}}
\newcommand{\CommentVarTok}[1]{\textcolor[rgb]{0.37,0.37,0.37}{\textit{#1}}}
\newcommand{\ConstantTok}[1]{\textcolor[rgb]{0.56,0.35,0.01}{#1}}
\newcommand{\ControlFlowTok}[1]{\textcolor[rgb]{0.00,0.23,0.31}{\textbf{#1}}}
\newcommand{\DataTypeTok}[1]{\textcolor[rgb]{0.68,0.00,0.00}{#1}}
\newcommand{\DecValTok}[1]{\textcolor[rgb]{0.68,0.00,0.00}{#1}}
\newcommand{\DocumentationTok}[1]{\textcolor[rgb]{0.37,0.37,0.37}{\textit{#1}}}
\newcommand{\ErrorTok}[1]{\textcolor[rgb]{0.68,0.00,0.00}{#1}}
\newcommand{\ExtensionTok}[1]{\textcolor[rgb]{0.00,0.23,0.31}{#1}}
\newcommand{\FloatTok}[1]{\textcolor[rgb]{0.68,0.00,0.00}{#1}}
\newcommand{\FunctionTok}[1]{\textcolor[rgb]{0.28,0.35,0.67}{#1}}
\newcommand{\ImportTok}[1]{\textcolor[rgb]{0.00,0.46,0.62}{#1}}
\newcommand{\InformationTok}[1]{\textcolor[rgb]{0.37,0.37,0.37}{#1}}
\newcommand{\KeywordTok}[1]{\textcolor[rgb]{0.00,0.23,0.31}{\textbf{#1}}}
\newcommand{\NormalTok}[1]{\textcolor[rgb]{0.00,0.23,0.31}{#1}}
\newcommand{\OperatorTok}[1]{\textcolor[rgb]{0.37,0.37,0.37}{#1}}
\newcommand{\OtherTok}[1]{\textcolor[rgb]{0.00,0.23,0.31}{#1}}
\newcommand{\PreprocessorTok}[1]{\textcolor[rgb]{0.68,0.00,0.00}{#1}}
\newcommand{\RegionMarkerTok}[1]{\textcolor[rgb]{0.00,0.23,0.31}{#1}}
\newcommand{\SpecialCharTok}[1]{\textcolor[rgb]{0.37,0.37,0.37}{#1}}
\newcommand{\SpecialStringTok}[1]{\textcolor[rgb]{0.13,0.47,0.30}{#1}}
\newcommand{\StringTok}[1]{\textcolor[rgb]{0.13,0.47,0.30}{#1}}
\newcommand{\VariableTok}[1]{\textcolor[rgb]{0.07,0.07,0.07}{#1}}
\newcommand{\VerbatimStringTok}[1]{\textcolor[rgb]{0.13,0.47,0.30}{#1}}
\newcommand{\WarningTok}[1]{\textcolor[rgb]{0.37,0.37,0.37}{\textit{#1}}}

\usepackage{longtable,booktabs,array}
\usepackage{calc} % for calculating minipage widths
% Correct order of tables after \paragraph or \subparagraph
\usepackage{etoolbox}
\makeatletter
\patchcmd\longtable{\par}{\if@noskipsec\mbox{}\fi\par}{}{}
\makeatother
% Allow footnotes in longtable head/foot
\IfFileExists{footnotehyper.sty}{\usepackage{footnotehyper}}{\usepackage{footnote}}
\makesavenoteenv{longtable}
\usepackage{graphicx}
\makeatletter
\newsavebox\pandoc@box
\newcommand*\pandocbounded[1]{% scales image to fit in text height/width
  \sbox\pandoc@box{#1}%
  \Gscale@div\@tempa{\textheight}{\dimexpr\ht\pandoc@box+\dp\pandoc@box\relax}%
  \Gscale@div\@tempb{\linewidth}{\wd\pandoc@box}%
  \ifdim\@tempb\p@<\@tempa\p@\let\@tempa\@tempb\fi% select the smaller of both
  \ifdim\@tempa\p@<\p@\scalebox{\@tempa}{\usebox\pandoc@box}%
  \else\usebox{\pandoc@box}%
  \fi%
}
% Set default figure placement to htbp
\def\fps@figure{htbp}
\makeatother





\setlength{\emergencystretch}{3em} % prevent overfull lines

\providecommand{\tightlist}{%
  \setlength{\itemsep}{0pt}\setlength{\parskip}{0pt}}



 


\KOMAoption{captions}{tableheading}
\makeatletter
\@ifpackageloaded{caption}{}{\usepackage{caption}}
\AtBeginDocument{%
\ifdefined\contentsname
  \renewcommand*\contentsname{Table of contents}
\else
  \newcommand\contentsname{Table of contents}
\fi
\ifdefined\listfigurename
  \renewcommand*\listfigurename{List of Figures}
\else
  \newcommand\listfigurename{List of Figures}
\fi
\ifdefined\listtablename
  \renewcommand*\listtablename{List of Tables}
\else
  \newcommand\listtablename{List of Tables}
\fi
\ifdefined\figurename
  \renewcommand*\figurename{Figure}
\else
  \newcommand\figurename{Figure}
\fi
\ifdefined\tablename
  \renewcommand*\tablename{Table}
\else
  \newcommand\tablename{Table}
\fi
}
\@ifpackageloaded{float}{}{\usepackage{float}}
\floatstyle{ruled}
\@ifundefined{c@chapter}{\newfloat{codelisting}{h}{lop}}{\newfloat{codelisting}{h}{lop}[chapter]}
\floatname{codelisting}{Listing}
\newcommand*\listoflistings{\listof{codelisting}{List of Listings}}
\makeatother
\makeatletter
\makeatother
\makeatletter
\@ifpackageloaded{caption}{}{\usepackage{caption}}
\@ifpackageloaded{subcaption}{}{\usepackage{subcaption}}
\makeatother
\usepackage{bookmark}
\IfFileExists{xurl.sty}{\usepackage{xurl}}{} % add URL line breaks if available
\urlstyle{same}
\hypersetup{
  pdftitle={Editing the R Markdown file},
  pdfauthor={Dr.~Karim Anaya-Izquierdo},
  colorlinks=true,
  linkcolor={blue},
  filecolor={Maroon},
  citecolor={Blue},
  urlcolor={Blue},
  pdfcreator={LaTeX via pandoc}}


\title{Editing the R Markdown file}
\author{Dr.~Karim Anaya-Izquierdo}
\date{}
\begin{document}
\maketitle


\subsubsection{Changing the header and adding
text}\label{changing-the-header-and-adding-text}

At the top of the file you find the header, which looks something like:

\begin{Shaded}
\begin{Highlighting}[]
\SpecialCharTok{{-}{-}{-}}
\NormalTok{title}\SpecialCharTok{:} \StringTok{"My title"}
\NormalTok{author}\SpecialCharTok{:} \StringTok{"Christian Rohrbeck"}
\NormalTok{date}\SpecialCharTok{:} \StringTok{"07/07/2023"}
\NormalTok{format}\SpecialCharTok{:}\NormalTok{ pdf}
\SpecialCharTok{{-}{-}{-}}
\end{Highlighting}
\end{Shaded}

This is where you can edit the header (title, author and date) of your
document. You do not have to change the line starting with ``output'' to
knit to HTML or Word.

Section headings are specified using hashes, such as

\begin{Shaded}
\begin{Highlighting}[]
\CommentTok{\# R Markdown}
\end{Highlighting}
\end{Shaded}

with \# being used for top level sections and \#\# for sub-sections,
should you want to use them.

To add text to the various sections of your document, you just have to
write it like in a Word document. The difference to Word is that you
will not see the layout until you knit to your output file. In RStudio
you will see the buttons ``Source'' and ``Visual'' to the upper left of
your file. By switching to ``Visual'' you will have a limited preview.

If you want to start a new paragraph, you have to leave a blank line -
it's not enough to just start a new line.

You can use LaTeX in R Markdown documents to display equations. For
example to show \(\sqrt{X}\) you can use \$\(\backslash\)sqrt\{X\}\$.
For more details on using LaTeX equations in R Markdown, see
\href{https://www.math.mcgill.ca/yyang/regression/RMarkdown/example.html}{here}.

\subsubsection{Inserting R code}\label{inserting-r-code}

Between the different lines of text (or paragraphs), we want to insert
bits of R code, for instance, to create a plot that is inserted at this
point of the document.

We can insert R code chunks using the keyboard short cut
\(\mathrm{\texttt{Ctrl + Alt + I}}\)
(\(\mathrm{\texttt{Cmd + Option + I }}\) on macOS). This will produce an
R chunk that is presented as

\texttt{```$\left\{\mathrm{r}\right\}$}\\
~\\
\texttt{```}

You can also type this directly into the R Markdown file.

Everything that is typed within this chunk will be evaluated using R.
For instance, we can generate two samples from a Normal(1,4)
distribution using

\texttt{```$\left\{\mathrm{r}\right\}$}\\
\texttt{rnorm( 2, mean=1, sd=2 )}\\
\texttt{```}

\begin{verbatim}
[1] 2.14651428 0.06004377
\end{verbatim}

\textbf{Tip:} You will find a small green play button in the right upper
corner of each R code chunk. You can use it to test the R chunk
individually and to verify that your code is working correctly.

\subsubsection{Loading R packages}\label{loading-r-packages}

It is important to remember that R Markdown ignores any packages or
variables loaded within your RStudio workspace when knitting to the
PDF/HTML/Word files. Hence you have to load them separately within the
document. This ensures reproducibility of your results.

When loading R packages, it is best to suppress any messages because
some messages may cause issues when knitting. To do this, we use the
options \textbf{warning=FALSE} and \textbf{message=FALSE} when loading
packages

\texttt{```$\left\{\mathrm{r, warning=FALSE, message=FALSE}\right\}$}\\
\texttt{library(dplyr)}\\
\texttt{```}

Please also remember that any packages you load must be installed on
your device.

\subsubsection{Figures in R Markdown}\label{figures-in-r-markdown}

By putting the code for producing our data graphics into the R Markdown
file, the figures are automatically generated and inserted into the
knitted text document. However, the data graphics may be quite large,
which is troubling when we have to stick to a page limit or are
concerned about presentation.

There are multiple options we can use to control the size and position
of the data graphic within the text document:

\begin{itemize}
\item
  \(\mathrm{\texttt{out.width}}\): Controls the width of the data
  graphic, relative to the width of the text.
\item
  \(\mathrm{\texttt{fig.height}}\) and \(\mathrm{\texttt{fig.width}}\):
  Width and height of the plot - the size of the plot may still be
  changed using \(\mathrm{\texttt{out.width}}\)
\item
  \(\mathrm{\texttt{fig.align}}\): Specifies alignment of the figure.
  Possible values are `default', `left', `right', and `center'.
\item
  \(\mathrm{\texttt{fig.cap}}\): Gives your graphic a caption.
\item
  \(\mathrm{\texttt{warning=FALSE}}\) and
  \(\mathrm{\texttt{message=FALSE}}\) to suppress R warnings and
  messages. \textbf{Only use this to save space in the final document
  and do not use from the start.}
\end{itemize}

There are many more chunk options, which you can find in the R Markdown
documentation. For the purposes of this course, the options above are
the important ones.

\textbf{Important:} Please remember that you cannot have a line break
when you specify options, no matter how many you set.

\textbf{Example:} Let's start by generating some random data samples in
one chunk of R code

This piece of R code does not produce any output because we stored
everything as variables, and the generated data is stored in the data
frame \textbf{obs}.

To visualize the data, we create a scatter plot of the two samples in a
new R chunk

When knitting the document, these two chunks of R code produce the
following output in the text document

\begin{Shaded}
\begin{Highlighting}[]
\FunctionTok{set.seed}\NormalTok{(}\DecValTok{2022}\NormalTok{)}
\NormalTok{x }\OtherTok{\textless{}{-}} \FunctionTok{rnorm}\NormalTok{( }\DecValTok{100}\NormalTok{ )}
\NormalTok{y }\OtherTok{\textless{}{-}} \FunctionTok{rnorm}\NormalTok{( }\DecValTok{100}\NormalTok{ )}
\NormalTok{obs }\OtherTok{\textless{}{-}} \FunctionTok{data.frame}\NormalTok{( }\StringTok{"x"}\OtherTok{=}\NormalTok{x, }\StringTok{"y"}\OtherTok{=}\NormalTok{y )}
\end{Highlighting}
\end{Shaded}

\begin{Shaded}
\begin{Highlighting}[]
\FunctionTok{plot}\NormalTok{( obs}\SpecialCharTok{$}\NormalTok{x, obs}\SpecialCharTok{$}\NormalTok{y, }\AttributeTok{xlab=}\StringTok{"Samples for X"}\NormalTok{, }\AttributeTok{ylab=}\StringTok{"Samples for Y"}\NormalTok{ )}
\end{Highlighting}
\end{Shaded}

\begin{figure}[H]

{\centering \includegraphics[width=0.4\linewidth,height=\textheight,keepaspectratio]{Editing-RMarkdown_files/figure-pdf/unnamed-chunk-5-1.pdf}

}

\caption{Scatter plot of X against Y.}

\end{figure}%




\end{document}
